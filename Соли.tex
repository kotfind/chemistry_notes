\newpage
\section{Соли}
\subsection{Классификация солей}
\begin{itemize}
    \item Средние (нормальные)\\
        $Na_3PO_4$, $K_2SO_4$

    \item Кислые\\
        $KHSO_4$, $NaH_2PO_4$

    \item Основные\\
        $Mg(OH)Cl$

    \item Комплексные\\
        $K_3[Fe(CN)_6]$

\end{itemize}


\subsection{Номенкулатура солей}
\noindent
\fbox{Название кислотного остатка + название металла}\\
\fbox{\textit{гидро-} если есть $H$}\\
\fbox{\textit{дигидро-} если есть $H_2$}\\
\fbox{\textit{гидроксо-} если есть $OH$}\\

%\noindent
%Примеры:\\
%$KHCO_3$ - \textit{гидро}карбонат калия.\\
%$NaH_2PO_4$ - \textit{дигидро}фосфат натрия.\\
%$AlOHCl_2$ - \textit{гидроксо}хлорид алюминия.\\

\begingroup
\renewcommand{\arraystretch}{1.5}
\begin{tabular}[b!]{| l | l | l | l |}
    \hline
    \multicolumn{4}{| c |}{\textbf{Названия солей некоторых кислот}}\\
    \hline
    \multicolumn{2}{| c |}{Кислота} & \multicolumn{2}{c |}{Кислотный остаток}\\
    \hline
    \multicolumn{1}{| c |}{Формула} & \multicolumn{1}{c |}{Название} & \multicolumn{1}{c}{Формула} & \multicolumn{1}{| c |}{Название}\\
    \hline
    $H_2SO_4$  & Серная     & $SO_4^{2-}$  & Сульфат\\
    $H_2SO_3$  & Сернистая  & $SO_3^{2-}$  & Сульфит\\
    $HNO_3$    & Азотная    & $NO_3^{-}$   & Нитрат\\
    $HNO_2$    & Азотистая  & $NO_2^{-}$   & Нитрит\\
    $H_3PO_4$  & Фосфорная  & $PO_4^{3-}$  & Фосфат\\
    $H_2CO_3$  & Угольная   & $CO_3^{2-}$  & Карбонат\\
    $H_2SiO_3$ & Кремниевая & $SiO_3^{2-}$ & Силикат\\
    \hline
    $HF$   & Фтороводородная (плавиковая) & $F^-$    & Фторид\\
    $HCl$  & Хлороводородная (соляная)    & $Cl^-$   & Хлорид\\
    $HBr$  & Бромоводородная              & $Br^-$   & Бромид\\
    $HI$   & Йодоводородная               & $I^-$    & Йодид\\
    $H_2S$ & Сероводородная               & $S^{2-}$ & Сульфид\\
    \hline
\end{tabular}
\endgroup


\subsection{Получение солей}
\begin{enumerate}
    \item Кислота + ...
    \begin{enumerate}
        \item К + акт. и полуакт. Ме\\
            $2H_3PO_4 + 6Na \ra 2Na_3PO_4 + 3H_2\ua$

        \item К + осн. оксид\\
            $3H_2SO_4 + Fe_2O_3 \tarrow Fe_2(SO_4)_3 + 3H_2O$

        \item К + основание\\
            $3HNO_3 + Cr(OH)_3 \ra Cr(NO_3)_3 + 3H_2O$
    \end{enumerate}

    \item Кислотный оксид + ...
    \begin{enumerate}
        \item Кисл. окс. + щелочь\\
            $N_2O_5 + Ca(OH)_2 \ra Ca(NO_3)_2 + H_2O$

        \item Кисл. окс. + осн. оксид\\
            $SiO_2 + CaO \tarrow CaSiO_3$
    \end{enumerate}

    \item Соль + ...
    \begin{enumerate}
        \item Соль + кислота\\
            $Ca_3(PO_4)_2 + 3H_2SO_4 \tarrow 3CaSO_4 + 2H_3PO_4$

        \item Соль + щелочь\\
            $Fe_2(SO_4)_3 + 6NaOH = 2Fe(OH)_3\da + 3Na_2SO_4$

        \item Соль + Ме\\
            $CuSO_4 + Fe \ra FeSO_4 + Cu\da$

        \item Соль + нелетуч. кис. оксид\\
            $CaCO_3 + SiO_2 \ra CaSiO_3 CO_2\ua$

        \item Соль + соль\\
            $Al_2(SO_4)_3 + 3BaCl_2 \ra 3BaSO_4\da + 2AlCl_3$
    \end{enumerate}

    \item Ме + ...
    \begin{enumerate}
        \item Ме + неМе\\
            $2Fe + 3Cl_2 \xrightarrow{горение} 2FeCl_3$
    \end{enumerate}

\end{enumerate}


\subsection{Химические свойства солей}
\begin{enumerate}
    \item Разложение некоторых солей при нагревании\\
        $CaCO_3 \tarrow CaO + CO_2\ua$

    \item С кислотами (кислота сильнее соли)\\
        $2NaCl + H_2SO_4 \ra Na_2SO_4 + HCl\ua$

    \item С щелочами (если соль растворима)\\
        $MgSO_4 + Ba(OH)_2 \ra BaSO_4\da + Mg(OH)_2$

    \item С солями (если обе соли растворимы, а хотя бы один продукт --- нет)\\
        $NaCl + AgNO_3 \ra AgCl + NaNO_3$

    \item C не акт. Ме, левее Ме соли\\
        $Fe + CuSO_4 \ra FeSO_4 + Cu\da$
\end{enumerate}
