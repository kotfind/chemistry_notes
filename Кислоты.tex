\newpage
\section{Кислоты}
\subsection{Классификация кислот}
\subsubsection{По содержанию кислорода}
\begin{itemize}
    \item Бескислородные\\
        $H_2S$, $HCl$, $HI$

    \item Кислородосодержащие\\
        $HClO_4$, $CH_3COOH$, $H_2SO_4$
\end{itemize}


\subsubsection{По числу атомов водорода}
\begin{itemize}
    \item Одноосновные\\
        $HCl$

    \item Многоосновные\\
        $H_2S$, $H_3PO_4$

\end{itemize}



\subsection{Номенкулатура кислот}
\subsubsection{Бескислородные}
\fbox{название элемента + "водородная"{}}\\
$HF$   --- фторо\textbf{водородная}\\
$HCl$  --- хлоро\textbf{водородная}\\
$H_2S$ --- серо\textbf{водородная}


\subsubsection{Кислородосодержащие}
\fbox{название элемента + суффикс + кислота}\\
Выбор суффикса зависит от степени окисления элемента.
Суффиксы в порядке уменьшения степени окисления:
\begin{enumerate}
    \item \textbf{-ная}, \textbf{-вая} (максимальная, соответствует номеру
                                        группы в таблице Менделеева)
    \item \textbf{-оватая}
    \item \textbf{-истая}
    \item \textbf{-оватистая}
\end{enumerate}
$HCl^{+7}O_4$ --- хлор\textbf{ная} кислота\\
$HCl^{+5}O_3$ --- хлор\textbf{новатая} кислота\\
$HCl^{+3}O_2$ --- хлор\textbf{истая} кислота\\
$HCl^{+1}O$   --- хлорн\textbf{оватистая} кислота\\



\subsection{Получение кислот}
\begin{enumerate}
    \item Бескислородные\\
        $H_2 + Cl \rightarrow 2HCl$\\
        $H_2 + S \rightarrow H_2S$

    \item Кислородные\\
        $H_2O + SO_3 \rightarrow H_2SO_4$\\
        $H_2O + CO_2 \rightarrow H_2CO_3$

    \item Универсальный (реагирует, если данная кислота левее, чем кислота соли в РАК)\\
        $CaCO_3 + 2HCl \rightarrow CaCl_2 + H_2CO_3$

\end{enumerate}


\subsection{Химические свойства кислот}
\begin{enumerate}
    \item Реакции с индикаторами (см. тему основания)

    \item С активными и средне активными Ме, образующими растворимые соли.\\
        \fbox{\fbox{$HNO_3$}} --- исключение.\\
        $Ba + HCl \rightarrow BaCl_2 + H_2\uparrow$\\
        $Ba + H_2SO_4 \not \rightarrow$ т.к. $BaSO_4$ --- нераст.\\
        $Au + HNO_3 \not \rightarrow$ т.к. $Au$ --- неакт. Ме.

    \item С основными и амфотерными оксидами\\
        $BaO + 2HNO3 \rightarrow Ba(NO_3)_2 + H_2O$

    \item С основаниями\\
        $2Fe(OH)_3 + 3H_2SO_4 \rightarrow H_2O + 6Fe_2(SO_4)_3$

    \item С солями (см. универсальный способ получения кислот)\\
        $K_2S + H_2SO_4 = K_2SO_4 + H_2S$

\end{enumerate}
