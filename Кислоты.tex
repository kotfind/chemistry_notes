\section{Кислоты}
\subsection{Классификация}
\subsubsection{По содержанию кислорода}
\begin{itemize}
    \item Бескислородные\\
        $H_2S$, $HCl$, $HI$

    \item Кислородосодержащие\\
        $HClO_4$, $CH_3COOH$, $H_2SO_4$
\end{itemize}


\subsubsection{По числу атомов водорода}
\begin{itemize}
    \item Одноосновные\\
        $HCl$

    \item Многоосновные\\
        $H_2S$, $H_3PO_4$

\end{itemize}



\subsection{Номенкулатура}
\subsubsection{Бескислородные}
\fbox{название элемента + "водородная"{}}\\
$HF$   --- фторо\textbf{водородная}\\
$HCl$  --- хлоро\textbf{водородная}\\
$H_2S$ --- серо\textbf{водородная}


\subsubsection{Кислородосодержащие}
\fbox{название элемента + суффикс + кислота}\\
Выбор суффикса зависит от степени окисления элемента.
Суффиксы в порядке уменьшения степени окисления:
\begin{enumerate}
    \item \textbf{-ная}, \textbf{-вая} (максимальная, соответствует номеру
                                        группы в таблице Менделеева)
    \item \textbf{-оватая}
    \item \textbf{-истая}
    \item \textbf{-оватистая}
\end{enumerate}
$HCl^{+7}O_4$ --- хлор\textbf{ная} кислота\\
$HCl^{+5}O_3$ --- хлор\textbf{новатая} кислота\\
$HCl^{+3}O_2$ --- хлор\textbf{истая} кислота\\
$HCl^{+1}O$   --- хлорн\textbf{оватистая} кислота\\
