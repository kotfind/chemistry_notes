\newpage
\section{Оксиды}
\subsection{Классификация оксидов}
\begin{itemize}
    \item Безразличные\\
    $CO$\\
    $NO$

    \item Солеобразующие
    \begin{itemize}
        \item Основные   \fbox{$МеO$(\rom{1}, \rom{2})}
        \item Амфотерные \fbox{$МеO$(\rom{3}) | \fbox{$BeO$, $ZnO$}}
        \item Кислотные  \fbox{$неМе$ | $МеO$(\rom{5} -- \rom{7})}
    \end{itemize}
\end{itemize}



\subsection{Получение оксидов}
\begin{enumerate}
    \item Окисление
    \begin{enumerate}
        \item Простых\\
            $S + 0_2 \ra SO_2\ua$
        \item Сложных\\
            $2H_2S + 3O_2 \ra 2H_2O + 2SO_2\ua$
    \end{enumerate}

    \item Разложение сложных веществ
    \begin{enumerate}
        \item Некоторых солей
        \item Некоторых кислот
        \item Всех нерастворимых оснований\\
            \fbox{$Е(OH) \tarrow ЕO + H_2O$}
    \end{enumerate}
\end{enumerate}



\subsection{Химические свойства оксидов}
\subsubsection{Основные}
\begin{enumerate}
    \item Вода (если $Ме$ --- активный)\\
        \fbox{основный оксид + вода $\ra$ основание}\\
        $CaO + H_2O \ra Ca(OH)_2$

    \item Кислоты\\
        \fbox{основный оксид + кислота $\ra$ соль + вода}\\
        $CuO + H_2SO_4 \ra CuSO_4 + H_2O$

    \item Кислотные оксиды\\
        \fbox{основный оксид + кислотный оксид $\ra$ соль}\\
        $CaO + Al_2O_3 \tarrow Ca(AlO_2)_2$

\end{enumerate}


\subsubsection{Кислотные}
\begin{enumerate}
    \item Вода\\
        %\fbox{$EO + H_2O \ra H(EO)$}\\
        \fbox{кислотный оксид + вода $\ra$ кислота}\\
        $SO_3 + H_2O \ra H_2SO_4$

    \item Щелочь\\
        \fbox{кислотный оксид + щелочь $\ra$ соль + вода}\\
        \fbox{\fbox{$SiO_2 + H_2O \not \ra$}}\\
        \fbox{\fbox{$P_2O_5 + H_2O \frac{\ra HPO_3}
                                        {\ra H_3PO_4}$}}\\
        $SO_2 + 2NaOH \ra Na_2SO_3 + H_2O$

    \item Основные оксиды\\
        \fbox{кислотный оксид + основный оксид $\ra$ соль}\\
        $CO_2 + CaO \ra CaCO_3$
\end{enumerate}
