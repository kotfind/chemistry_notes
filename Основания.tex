\section{Основания}
\subsection{Классификация}
\begin{itemize}
    \item Щелочи (растворимые в воде)
    \item Нерастворимые (в воде)
\end{itemize}



\subsection{Получение}
\subsubsection{Щелочи}
\begin{enumerate}
    \item Вода с активными Ме\\
        $2Na + 2H_2O \rightarrow 2NaOH + H_2\uparrow$\\
        $Ca + 2H_2O \rightarrow Ca(OH)_2 + H_2\uparrow$

    \item Вода с оксидами активных Ме\\
        $Li_2O + H_2O \rightarrow 2LiOH$\\
        $CaO + H_2O \rightarrow Ca(OH)_2$

    \item Электролиз раствора хлорида натрия или калия\\
        $2NaCl + 2H_2O \stackrel{эл. ток}{\rightarrow} 2NaOH + H_2\uparrow + Cl_2\uparrow$\\
        $2KCl + 2H_2O \stackrel{эл. ток}{\rightarrow} 2KOH + H_2\uparrow + Cl_2\uparrow$
        
    \item \fbox{соль + щелочь $\rightarrow$ соль + щелочь}\\
        $K_2SO_4 + Ba(OH)_2 = BaSO_4\downarrow + KOH$
\end{enumerate}


\subsubsection{Нерастворимые}
\begin{enumerate}
    \item Раствор соли и раствор щелочи\\
        $CuCl_2 + 2KOH \rightarrow Cu(OH)_2\downarrow + 2KCl$\\
        $FeCl_3 + 3NaOH \rightarrow Fe(OH)_3\downarrow + 3NaCl$
\end{enumerate}



\subsection{Химические свойства}
\subsection{Щелочи}
\begin{enumerate}
    \item Изменение окраски индикаторов растворами щелочей
        \begin{figure*}[h!]
            \centering
            \begin{tabular}[b]{| l | c | c | c |}
                \hline
                Индикатор\textbackslash Среда & Нейтральная & Кислая & Щелочная \\
                \hline
                Лакмус & фиолетовый & красный & синий \\
                \hline
                Фенолфталеин & --- & --- & малиновый \\
                \hline
                Метилоранж & оранжевый & розовый & желтый \\
                \hline
            \end{tabular}
            \caption{Изменение окраски индикаторов растворами щелочей}
        \end{figure*}

    \item Кислоты (Нейтрализация)\\
        $NaOH + HCl \rightarrow NaCl + H_2O$\\
        $Ca(OH)_2 + 2HNO_3 \rightarrow Ca(NO_3)_2 + 2H_2O$

    \item Кислотные оксиды\\
        $Ca(OH)_2 + CO_2 \rightarrow CaCO_3\downarrow + H_2O$\\
        $2KOH + CO_2 \rightarrow K_2CO_3 + H_2O$

    \item Растворимые в воде соли\\
        $HCl_3 + 3KOH \rightarrow Al(OH)_3 + 3KCl$

\end{enumerate}


\subsection{Нерастворимые}
\begin{enumerate}
    \item Термическое разложение \fbox{основание $\tarrow$ основный оксид + вода}\\
        $2Al(OH)_3 \tarrow Al_2O_3 + 3H_2O$

    \item Кислоты (Нейтрализация)\\
        $Cu(OH)_2 + 2HNO_3 \rightarrow Cu(NO_3)_2 + 2H_2O$
\end{enumerate}
