\section{Основания}
\subsection{Классификация}
\begin{itemize}
    \item Щелочи (растворимые в воде)
    \item Нерастворимые (в воде)
\end{itemize}



\subsection{Получение}
\subsubsection{Щелочи}
\begin{enumerate}
    \item Вода с активными Ме\\
        $2Na + 2H_2O \rightarrow 2NaOH + H_2\uparrow$\\
        $Ca + 2H_2O \rightarrow Ca(OH)_2 + H_2\uparrow$

    \item Вода с оксидами активных Ме\\
        $Li_2O + H_2O \rightarrow 2LiOH$\\
        $CaO + H_2O \rightarrow Ca(OH)_2$

    \item Электролиз раствора хлорида натрия или калия\\
        $2NaCl + 2H_2O \stackrel{эл. ток}{\rightarrow} 2NaOH + H_2\uparrow + Cl_2\uparrow$\\
        $2KCl + 2H_2O \stackrel{эл. ток}{\rightarrow} 2KOH + H_2\uparrow + Cl_2\uparrow$
        
    \item \fbox{соль + щелочь $\rightarrow$ соль + щелочь}\\
        $K_2SO_4 + Ba(OH)_2 = BaSO_4\downarrow + KOH$
\end{enumerate}


\subsubsection{Нерастворимые}
\begin{enumerate}
    \item Раствор соли и раствор щелочи\\
        $CuCl_2 + 2KOH \rightarrow Cu(OH)_2\downarrow + 2KCl$\\
        $FeCl_3 + 3NaOH \rightarrow Fe(OH)_3\downarrow + 3NaCl$
\end{enumerate}
